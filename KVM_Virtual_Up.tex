\documentclass[12pt, twoside, openany]{book}

% 设置页边距
\usepackage[a4paper,top=2.5cm,bottom=2.5cm,left=2.5cm,right=2cm,headheight=15pt]{geometry}

% 中文支持
\usepackage[UTF8]{ctex}

% 其他常用包
\usepackage{graphicx} % 插入图片
\usepackage{amsmath, amssymb} % 数学公式
\usepackage{amsthm} % 定理环境
\usepackage[hidelinks]{hyperref} % 超链接
\usepackage{bookmark} % 更好地处理PDF书签和目录
%\usepackage{titlesec} % 标题格式 - 注释掉以防干扰自定义样式
\usepackage{titletoc} % 目录格式
\usepackage[nottoc]{tocbibind} % 目录和附录
\usepackage{fancyhdr} % 页眉页脚
\usepackage[table,dvipsnames]{xcolor} % 支持颜色,允许表格和tcolorbox使用颜色名
\definecolor{gray}{rgb}{0.9,0.9,0.9} % 定义gray颜色
\usepackage[most]{tcolorbox} % 用于创建精美的彩色文本框
\usepackage{etoolbox} % 用于修改标题格式
\usepackage{tikz} % 用于绘制斜线装饰
\usepackage{listings} % 用于代码高亮
\usepackage[format=hang,font=small,textfont=it]{caption}
\usepackage{natbib} % For bibliography support


\lstset{
    basicstyle          =   \ttfamily,          % 基本代码风格
    keywordstyle        =   \bfseries,          % 关键字风格
    commentstyle        =   \rmfamily\itshape,  % 注释的风格,斜体
    stringstyle         =   \ttfamily,  % 字符串风格
    flexiblecolumns,                % 别问为什么,加上这个
    numbers             =   left,   % 行号的位置在左边
    showspaces          =   false,  % 是否显示空格,显示了有点乱,所以不现实了
    numberstyle         =   \zihao{-5}\ttfamily,    % 行号的样式,小五号,tt等宽字体
    showstringspaces    =   false,
    captionpos          =   t,      % 这段代码的名字所呈现的位置,t指的是top上面
    frame               =   lrtb,   % 显示边框
    breaklines          =   true    %  行代码换行
}

\renewcommand{\lstlistingname}{\it 代码} % 代码的标题

% 经验总结环境
\newcounter{expnum}[chapter]
\setcounter{expnum}{0} % 确保从0开始,这样第一次使用时为1
\renewcommand{\theexpnum}{\thechapter.\arabic{expnum}} % 设置经验总结编号格式

% 使用一个更简单但可靠的方式定义经验总结环境
\newtcolorbox{experiencebox}[2][]{%
  enhanced,
  breakable,
  colback=gray!100,
  colframe=black,
  arc=2mm,
  boxrule=1pt,
  left=10pt,
  right=10pt,
  top=10pt,
  bottom=5pt,
  before upper={\stepcounter{expnum}\noindent\textbf{【经验总结 \theexpnum】}~#2},
  #1
}

\newtheoremstyle{brack}
{6pt} % Space above
{6pt} % Space below
{\normalfont \kaishu} % Body font
{0pt} % Indent amount
{\bfseries} % Theorem head font
{} % Punctuation after theorem head
{0.5em} % Space after theorem head
{【#1 #2】} % Theorem head spec (can be left empty, meaning `normal')
\theoremstyle{brack}

\newtheorem{lab}{实验}[chapter]
\newtheorem{change}{思考}[lab]


% 使用TikZ创建章节标题样式
\usetikzlibrary{shapes.geometric,calc}

% 定义章节标题TikZ命令
\newcommand{\chaptertikztitle}[1]{%
  \begin{tikzpicture}[baseline]
    % 创建左侧图形组
    \begin{scope}
      % 灰色圆环 - 调整尺寸
      \fill[gray!100] (0,0) circle (1cm);
      \fill[white] (0,0) circle (0.65cm);
      
      % 外部虚线圆 - 更深的灰色
      \draw[dashed, black!50, line width=2pt] (0,0) circle (1.25cm);      % 单个平行四边形(简化设计)
      \begin{scope}[shift={(0,0)}]


        
        % 文字
        \node[font=\Large\bfseries, align=center] at (0,0) {第\arabic{chapter} 章};
      \end{scope}
    \end{scope}    
    % 添加右侧的标题文字 - 更大字体
    \node[anchor=west, font=\LARGE\bfseries] at (2.2,0) {#1};
  \end{tikzpicture}%
}

% 自定义章节标题格式
\makeatletter
\newcommand{\chaptertitle}[1]{%
  \chaptertikztitle{#1}%
}

% 替换带编号的章节标题格式
\def\@makechapterhead#1{%
  \vspace*{30\p@}%
  {\parindent \z@ \normalfont
    \ifnum \c@secnumdepth >\m@ne
      \if@mainmatter
        \chaptertitle{#1}\par\nobreak
      \else
        % 前言部分使用普通格式
        \begin{tikzpicture}
          \node[anchor=west, inner sep=0pt] at (0,0) {\Huge\bfseries #1};
          \draw[dashed, line width=1pt, gray] (0,-0.85) -- (\textwidth,-0.85);
        \end{tikzpicture}\par\nobreak
      \fi
    \else
      % 无编号章节的处理
      \begin{tikzpicture}
        \node[anchor=west, inner sep=0pt] at (0,0) {\Huge\bfseries #1};
        \draw[dashed, line width=1pt, gray] (0,-0.85) -- (\textwidth,-0.85);
      \end{tikzpicture}\par\nobreak
    \fi
    \vskip 40\p@
  }}

% 同样应用样式到无编号章节  
\def\@makeschapterhead#1{%
  \vspace*{30\p@}%
  {\parindent \z@ \centering
    \normalfont
    \begin{tikzpicture}
    % 无编号章节只显示标题和虚线,居中显示
    \node[anchor=center, inner sep=0pt] at (0,0) {\Huge\bfseries #1};
    \draw[dashed, line width=1pt, gray] (-0.5\textwidth,-0.85) -- (0.5\textwidth,-0.85);
    \end{tikzpicture}\par\nobreak
    \vskip 40\p@
  }}
\makeatother

% 章节和小节标题样式
% 小节标题使用传统的方式居中
\makeatletter
\let\oldsection\section
\renewcommand{\section}[2][\relax]{%
  \ifx#1\relax
    \oldsection[#2]{\centering #2}%
    \sectionmark{#2}% Override the mark with non-centered text
  \else
    \oldsection[#1]{\centering #2}%
  \fi
}
\makeatother

\renewcommand{\chaptermark}[1]{\markboth{第\thechapter 章 \ #1}{}}

% 自定义章节目录格式
\titlecontents{chapter}[2em]{\bfseries\large}{\ifnum\thecontentslabel>0 第\,\thecontentslabel\,章\quad\else\fi}{\hspace*{-1.5em}}{\hfill\contentspage}
\titlecontents{section}[7em]{\normalfont}{{\contentslabel{2.5em}}}{\hspace*{-2.5em}}{\titlerule*[0.5em]{.}\contentspage}
\titlecontents{subsection}[9.6em]{\normalfont}{{\contentslabel{2.5em}}}{\hspace*{-2.5em}}{\titlerule*[0.5em]{.}\contentspage}

% 简单的页眉设置
\fancypagestyle{mainStyle}{%
  \fancyhf{} % 清除所有页眉页脚
  \fancyhead[RO]{\it \leftmark} % 奇数页右侧显示章标题
  \fancyhead[LE]{\it \rightmark} % 偶数页左侧显示节标题
  \renewcommand{\headrulewidth}{0pt} % 移除页眉下方的横线

  
  % 页码位置
  \fancyfoot[LE,RO]{\thepage}
}

\begin{document}
\frontmatter
\title{\heiti \textbf{KVM}虚拟化进阶与提高}
\author{\kaishu 作者:李杰}
\date{\empty}
\maketitle


\setcounter{page}{1}
\pagestyle{plain}


\chapter*{前\hspace{2em}言}
该内容为个人学习KVM虚拟化进阶与提高过程中所记述的笔记,内容主要包括虚拟网络高级特性、虚拟机迁移、KVM群集、嵌套虚拟化、性能监视与优化、P2V、V2V迁移、备份与恢复、oVirt(RHEV)安装和基本管理等方面的内容。

\textbf{虚拟网络高级特性包括}:多物理网卡的绑定、配置VLAN、网络过滤。\textbf{虚拟机迁移}:静态迁移和动态迁移。\textbf{KVM群集三种方式}:NFS、GFS2、OCFS2。\textbf{嵌套虚拟化}:KVM虚拟机中运行KVM虚拟机。\textbf{性能监视与优化}:CPU、内存、IO、网络等性能监视与优化。\textbf{P2V}:物理机转虚拟机。\textbf{V2V}:虚拟机转虚拟机。\textbf{备份与恢复}:快照和备份。\textbf{oVirt(RHEV)}安装和基本管理。

\begin{flushright}
\kaishu 李杰\\
\today
\end{flushright}

\chapter*{术\hspace{1.8em}语\hspace{1.8em}表}
\begin{description}
\item[KVM] 基于内核的虚拟机(Kernel-based Virtual Machine),是Linux内核中的一个模块,允许在Linux系统上运行虚拟机。
\item[VLAN] 虚拟局域网(Virtual Local Area Network),是一种在单一物理网络上创建多个逻辑网络的技术。
\item[oVirt] 红帽企业虚拟化管理工具,提供了一个集中管理虚拟化基础设施的平台。
\item[P2V] 物理机转虚拟机(Physical to Virtual),将物理服务器转换为虚拟机的过程。
\item[V2V] 虚拟机转虚拟机(Virtual to Virtual),将一种虚拟化平台上的虚拟机转换到另一种虚拟化平台的过程。
\end{description}


% 自定义目录标题格式 - 分散显示
\renewcommand{\contentsname}{\centerline{\Huge 目\hspace{2em}录}}
\tableofcontents




\mainmatter
\pagestyle{mainStyle} % 使用新定义的页面样式
\renewcommand{\chaptermark}[1]{\markboth{第\thechapter 章~~~#1}{}}
\renewcommand{\sectionmark}[1]{\markright{\thesection~~~#1}}

\chapter{虚拟网络的高级特性}
\section{多物理网卡的绑定}
如果一台服务器只有一块物理网卡,那么虚拟机只能使用这块网卡,无法使用其他的网卡。为了提高网络性能(提供负载平衡与冗余),可以将多块物理网卡绑定在一起,形成一个虚拟网卡,这样虚拟机就可以使用这个虚拟网卡。

\subsection{配置多网卡绑定的KVM桥接模式}
接下来将介绍如何配置多网卡绑定的KVM桥接模式。假设有两块物理网卡eth0和eth1,分别连接到不同的交换机上。我们将这两块网卡绑定在一起,形成一个虚拟网卡bond0,然后将bond0配置为KVM桥接模式。
\begin{lab}
  (配置多网卡绑定的KVM桥接模式)
  \begin{enumerate}
    \item 绑定网卡
    \begin{enumerate}
      \item 启用bonding模块
      \item 配置物理网卡
      \item 配置绑定接口
      \item 重新启动服务
      \item 测试
    \end{enumerate}
    \item 配置桥接
  \end{enumerate}
\end{lab}
首先使用\lstinline|ip link、ifconfig|指令查看当前的网络配置,确认与eth0和eth1相关的信息。接在使用virtsh或者virt-manager新建网络接口,选择桥接模式,网卡选择bond0。接着使用\lstinline|ifconfig bond0|指令查看bond0的状态,确认bond0已经创建成功。
接着使用\lstinline|modprobe bonding|指令启用bonding模块。使用\lstinline|lsmod grep bonding|指令查看当前加载的模块,确认bonding模块已经加载成功。
在主网卡eth0和eth1的配置文件中添加以下内容(提前做好备份):
\begin{lstlisting}[caption={网卡配置文件}]
  # ifcfg-eth0
  {
    Type=Ethernet
    BOOTPROTO=none
    NAME=eth0
    DEVICE=eth0
    ONBOOT=yes
    MASTER=bond0 # 绑定到bond0
    SLAVE=yes
    NM_CONTROLLED=no # 不受NetworkManager控制
    USERCTL=no # 不允许用户控制
  }
  # ifcfg-eth1
  {
    Type=Ethernet
    BOOTPROTO=none
    NAME=eth1
    DEVICE=eth1
    ONBOOT=yes
    MASTER=bond0
    SLAVE=yes
    NM_CONTROLLED=no
    USERCTL=no
  }
  # ifcfg-bond0
  {
    DEVICE=bond0
    ONBOOT=yes
    NM_CONTROLLED=no
    USERCTL=no
    BONDING_OPTS="mode=1 miimon=100"
    BOOTPROTO=static
    IPADDR=192.168.200.11
    NETMASK=255.255.255.0
  }
\end{lstlisting}

以上只有TYPE、BOOTPROTO、DEVICE、ONBOOT、MASTER、SLAVE、NM\_CONTROLLED、USERCTL、BONDING\_OPTS、BOOTPROTO、IPADDR、NETMASK等参数是必须的,其他参数可以根据需要添加。接着使用\lstinline|ifup bond0|指令启用bond0接口,使用\lstinline|ifconfig bond0|指令查看bond0的状态,确认bond0已经创建成功。
\begin{change}
对于mode有七种选项,分别是
\begin{itemize}
  \item \textbf{mode=0 (balance-rr)}: 轮询模式,数据包按顺序在每个接口上发送,提供负载均衡和容错能力。
  \item \textbf{mode=1 (active-backup)}: 主备模式,只有一个接口处于活动状态,其他接口作为备份,提供高可用性。
  \item \textbf{mode=2 (balance-xor)}: 基于源MAC地址和目标MAC地址的哈希算法选择接口,提供负载均衡和容错能力。
  \item \textbf{mode=3 (broadcast)}: 广播模式,所有数据包在所有接口上发送,提供容错能力。
  \item \textbf{mode=4 (802.3ad)}: IEEE 802.3ad 动态链路聚合模式,需要交换机支持,提供负载均衡和高性能。
  \item \textbf{mode=5 (balance-tlb)}: 适配器传输负载均衡,根据负载动态分配流量,接收流量由当前接口处理。
  \item \textbf{mode=6 (balance-alb)}: 适配器负载均衡,扩展了 mode=5,支持接收和发送流量的负载均衡。
\end{itemize}
\end{change}







\end{document}
